\section{Ideen}
\label{sec:ideen}

\subsection{Welt}
\label{subsec:welt}
\begin{itemize}
	\item 2D-Welt
	\item verschiedene Welten in Form von Orbs (Spheren, Glaskugeln)
	\begin{itemize}
		\item Unterschiedliche Orbs beherrbergen unterscheidliche Monster\ref{subsec:monster}, Mineralien\ref{subsec:mineralien} und Biome\ref{subsec:biome}
		\item als Mittelpunkt des Universums könnte eine optische Täuschung in Form einer Spirale dienen (Sowas in der Art: http://www.neurowelt.de/Bilder-Vortrag/spirale2-25.jpg)
	\end{itemize}
	\item bestehend aus Klötzen (technische durch Quadtrees definiert)
	\item Mehr-Layer-Darstellung
	\begin{itemize}
		\item Vordergrund
			\subitem Ebene für Figuren und Ressourcen
		\item Gebäude-Layer
			\subitem Ebene für Gebäude und vielleicht zusätzliche Ressourcen
		\item Hintergrund
			\subitem nochmals unterteilt in ca. 3 Ebenen um einen Tiefeneffekt zu erzielen (wie bei Terraria)
		\item Ziel der Ebenen: man kann an Häusern vorbei laufen, ohne darüber springen oder durchlaufen zu müssen
	\end{itemize}
	\item Reisen von Orb zu Orb durch Teleportation
	\begin{itemize}
		\item Teleporter in jedem Orb
		\item Teleporter können transportiert und an anderen Stellen verankert werden
		\item Ressourcen für Teleporterbau befinden sich im Start-Orb
	\end{itemize}
	\item Orbs können physikalische Spielereien haben, wie z.B. andere Gravitation, vertauschte Steuerung. Spieler kann die Effekte umgehen, indem er die passenden Items angelegt, bzw. dabei hat oder durch Magie usw.
\end{itemize}


\subsection{Biome}
\label{subsec:biome}

\section{Charaktere}
\label{sec:charaktere}

\subsection{Charakter}
\label{subsec:charakter}

\subsection{NPCs}
\label{subsec:NPCs}

\subsection{Generationen}
\label{subsec:generationen}

\subsection{Klassen}
\label{subsec:klassen}

\begin{itemize}
	\item Nahkämpfer
	\item Magier
	\item Fernkämpfer
	\item Heiler
	\item eventuell Mischklassen und Unterklassen
\end{itemize}

\subsection{Rassen}
\label{subsec:rassen}

\subsection{Berufe}
\label{subsec:berufe}

\subsection{Monster}
\label{subsec:monster}
\begin{itemize}
	\item verschiedene AIs
\end{itemize}

\subsection{Bosse}
\label{subsec:bosse}

\section{Spielmechanik}
\label{sec:spielmechanik}

\subsection{Spielmodi}
\label{subsec:spielmodi}

\subsection{Multiplayer}
\label{subsec:multiplayer}

\subsection{Crafting-System}
\label{subsec:crafting}

\subsection{Städte}
\label{subsec:staedte}

\subsection{Fraktionen}
\label{subsec:Fraktionen}

\subsection{Handel}
\label{subsec:handel}

\subsection{Mineralien}
\label{subsec:mineralien}

\subsection{Magie}
\label{subsec:magie}

\subsection{Kampf-System}
\label{subsec:kamp-system}

\section{Weltgenerator}
\label{sec:weltgenerator}


    Rassen-System
        Mensch
        Ork
        Zwerg
        Elfen
        Gnome
        ...
    Berufs-System
        Mechaniker (kann Strom legen und verbinden usw.)
        Klempner (kann Wasserleitungen legen usw.) (auch für andere Flüssigkeiten)
    Dummies (Bots, Helferleins, andere Charaktere), die einem bei der Ressourcenbeschaffung und -Verarbeitung unterstützen
        Dummies können kämpfen (Berufs-System würde sich hier anbieten) (z.B. Militär-Dummies, die von Türmen aus schießen)
        Dummies sind NICHT für Aufgaben wie Schmied usw. da
        Dummies stellen Produktionsketten dar (z.B. Steinmetz hackt stein --> findet Erze -> bringt sie zum verarbeiten)
        Dummies werden genutzt um eine große Menge an Ressourcen zu sammeln, für spätere großen Projekte
    Magie-System
        da Magie übermächtig ist, muss man sie sich verdienen (lernen, studieren, forschen usw.)
    Generation-System (in Verbindung mit Klassen und Berufen)
        1. Generation hat 10 Punkte einer Klasse
        2. Generation hat 6 Punkte der Vater-Klasse und 6 Punkte der Mutter-Klasse (12 Punkte insgesamt)
        ...
    Charkater-Erstellung
        verschiedene Haare usw.
        Kleidung
        dick, dünn, groß, klein usw.

    NPC-Städte mit Politiksystem (einhergehend mit Fraktionen)
    NPC-Fraktionen => Verbündete, Neutrale, Feinde
    Der Spieler hat die Möglichkeit eine eigene Fraktion mit eigenen "Anhängern" zu gründen, die sich in der Welt behaupten muss
        Loyalitätsskalen
        Aggressorenskalen
    Der Spieler hat die Möglichkeit die "Art seiner Macht" vor Spielbeginn zu wählen
        Hexerei, Magie, Okkultismus, Religionen, Technik, Kinetik, Ying-Yang usw.
        Vielleicht begründend durch den Glauben an bestimmte Gottheiten? (Sacrifice)
    Spielermodi
        Softcore (Spieler verliert nichts oder nur Gold)
        Mediumcore (Spieler verliert einen Teil seiner Ausrüstung, kann wieder aufgesammelt werden)
        Hardcore (Spieler verliert seine Ausrüstung beim Sterben, kann nicht mehr aufgesammelt werden)
        Deathcore (Tot ist TOT!)
    Handelssystem
        Spieler kann Güter zum Tausch verwenden
        Es kann allerdings auch eine Währung eingesetzt werden
        Karawanen die regelmäßig für Gold sorgen, die allerdings auch beschützt werden müssen (Piraten, Räuber, Ronins usw.)

Fragen die es zu klären gilt:

    Gibt es eine Story?
    Bosse? Mini-, Mid- und Endbosse?
    Währungen?