\section{Welt}
\label{sec:welt}
	\begin{itemize}
		\item 2D-Welt
		\item verschiedene Welten in Form von Orbs (Spheren, Glaskugeln)
		\begin{itemize}
			\item Unterschiedliche Orbs beherrbergen unterscheidliche Monster \ref{subsec:monster}, Ressourcen \ref{subsec:ressourcen} und Biome \ref{subsec:biome}
			\item als Mittelpunkt des Universums könnte eine optische Täuschung in Form einer Spirale dienen (Sowas in der Art: http://www.neurowelt.de/Bilder-Vortrag/spirale2-25.jpg)
		\end{itemize}
		\item bestehend aus Klötzen (technische durch Quadtrees definiert)
		\item Mehr-Layer-Darstellung
		\begin{itemize}
			\item Vordergrund
				\subitem Ebene für Figuren und Ressourcen
			\item Gebäude-Layer
				\subitem Ebene für Gebäude und vielleicht zusätzliche Ressourcen
			\item Hintergrund
				\subitem nochmals unterteilt in ca. 3 Ebenen um einen Tiefeneffekt zu erzielen (wie bei Terraria)
			\item Ziel der Ebenen: man kann an Häusern vorbei laufen, ohne darüber springen oder durchlaufen zu müssen
		\end{itemize}
		\item Reisen von Orb zu Orb durch Teleportation
		\begin{itemize}
			\item Teleporter in jedem Orb
			\item Teleporter können transportiert und an anderen Stellen verankert werden
			\item Ressourcen für Teleporterbau befinden sich im Start-Orb
		\end{itemize}
		\item Orbs können physikalische Spielereien haben, wie z.B. andere Gravitation, vertauschte Steuerung. Spieler kann die Effekte umgehen, indem er die passenden Items angelegt, bzw. dabei hat oder durch Magie usw.
	\end{itemize}

\subsection{Biome}
\label{subsec:biome}
%	\begin{itemize}
%		\item 
%	\end{itemize}